\documentclass[]{article}
\usepackage{lmodern}
\usepackage{amssymb,amsmath}
\usepackage{ifxetex,ifluatex}
\usepackage{fixltx2e} % provides \textsubscript
\ifnum 0\ifxetex 1\fi\ifluatex 1\fi=0 % if pdftex
  \usepackage[T1]{fontenc}
  \usepackage[utf8]{inputenc}
\else % if luatex or xelatex
  \ifxetex
    \usepackage{mathspec}
    \usepackage{xltxtra,xunicode}
  \else
    \usepackage{fontspec}
  \fi
  \defaultfontfeatures{Mapping=tex-text,Scale=MatchLowercase}
  \newcommand{\euro}{€}
\fi
% use upquote if available, for straight quotes in verbatim environments
\IfFileExists{upquote.sty}{\usepackage{upquote}}{}
% use microtype if available
\IfFileExists{microtype.sty}{%
\usepackage{microtype}
\UseMicrotypeSet[protrusion]{basicmath} % disable protrusion for tt fonts
}{}
\usepackage[margin=1in]{geometry}
\usepackage{color}
\usepackage{fancyvrb}
\newcommand{\VerbBar}{|}
\newcommand{\VERB}{\Verb[commandchars=\\\{\}]}
\DefineVerbatimEnvironment{Highlighting}{Verbatim}{commandchars=\\\{\}}
% Add ',fontsize=\small' for more characters per line
\usepackage{framed}
\definecolor{shadecolor}{RGB}{248,248,248}
\newenvironment{Shaded}{\begin{snugshade}}{\end{snugshade}}
\newcommand{\KeywordTok}[1]{\textcolor[rgb]{0.13,0.29,0.53}{\textbf{{#1}}}}
\newcommand{\DataTypeTok}[1]{\textcolor[rgb]{0.13,0.29,0.53}{{#1}}}
\newcommand{\DecValTok}[1]{\textcolor[rgb]{0.00,0.00,0.81}{{#1}}}
\newcommand{\BaseNTok}[1]{\textcolor[rgb]{0.00,0.00,0.81}{{#1}}}
\newcommand{\FloatTok}[1]{\textcolor[rgb]{0.00,0.00,0.81}{{#1}}}
\newcommand{\CharTok}[1]{\textcolor[rgb]{0.31,0.60,0.02}{{#1}}}
\newcommand{\StringTok}[1]{\textcolor[rgb]{0.31,0.60,0.02}{{#1}}}
\newcommand{\CommentTok}[1]{\textcolor[rgb]{0.56,0.35,0.01}{\textit{{#1}}}}
\newcommand{\OtherTok}[1]{\textcolor[rgb]{0.56,0.35,0.01}{{#1}}}
\newcommand{\AlertTok}[1]{\textcolor[rgb]{0.94,0.16,0.16}{{#1}}}
\newcommand{\FunctionTok}[1]{\textcolor[rgb]{0.00,0.00,0.00}{{#1}}}
\newcommand{\RegionMarkerTok}[1]{{#1}}
\newcommand{\ErrorTok}[1]{\textbf{{#1}}}
\newcommand{\NormalTok}[1]{{#1}}
\usepackage{longtable,booktabs}
\ifxetex
  \usepackage[setpagesize=false, % page size defined by xetex
              unicode=false, % unicode breaks when used with xetex
              xetex]{hyperref}
\else
  \usepackage[unicode=true]{hyperref}
\fi
\hypersetup{breaklinks=true,
            bookmarks=true,
            pdfauthor={Max Wagner},
            pdftitle={Chapter 5: 5.6, 5.14, 5.20, 5.32, 5.48},
            colorlinks=true,
            citecolor=blue,
            urlcolor=blue,
            linkcolor=magenta,
            pdfborder={0 0 0}}
\urlstyle{same}  % don't use monospace font for urls
\setlength{\parindent}{0pt}
\setlength{\parskip}{6pt plus 2pt minus 1pt}
\setlength{\emergencystretch}{3em}  % prevent overfull lines
\setcounter{secnumdepth}{0}

%%% Use protect on footnotes to avoid problems with footnotes in titles
\let\rmarkdownfootnote\footnote%
\def\footnote{\protect\rmarkdownfootnote}

%%% Change title format to be more compact
\usepackage{titling}

% Create subtitle command for use in maketitle
\newcommand{\subtitle}[1]{
  \posttitle{
    \begin{center}\large#1\end{center}
    }
}

\setlength{\droptitle}{-2em}
  \title{Chapter 5: 5.6, 5.14, 5.20, 5.32, 5.48}
  \pretitle{\vspace{\droptitle}\centering\huge}
  \posttitle{\par}
  \author{Max Wagner}
  \preauthor{\centering\large\emph}
  \postauthor{\par}
  \predate{\centering\large\emph}
  \postdate{\par}
  \date{October 31, 2015}



\begin{document}

\maketitle


\begin{center}\rule{0.5\linewidth}{\linethickness}\end{center}

5.6

\begin{Shaded}
\begin{Highlighting}[]
\NormalTok{x <-}\StringTok{ }\DecValTok{77} \NormalTok{-}\StringTok{ }\DecValTok{65}
\NormalTok{x2 <-}\StringTok{ }\NormalTok{x /}\StringTok{ }\DecValTok{2}
\NormalTok{answer <-}\StringTok{ }\DecValTok{65} \NormalTok{+}\StringTok{ }\NormalTok{x2}
\end{Highlighting}
\end{Shaded}

The mean is the midpoint: 71

ME is: 6

DF are: 24

tdf: 1.71

forumla: ME = t(s/sqrt(n)) -\textgreater{} 6 = 1.71(s/sqrt(25))
-\textgreater{} s = 17.5439

\begin{center}\rule{0.5\linewidth}{\linethickness}\end{center}

5.14

\begin{enumerate}
\def\labelenumi{\alph{enumi}.}
\item
\end{enumerate}

SD = 250

ME = 25

(1 - CI / 2) = 0.05

z = 1.65

\begin{Shaded}
\begin{Highlighting}[]
\NormalTok{((}\FloatTok{1.65} \NormalTok{*}\StringTok{ }\DecValTok{250}\NormalTok{) /}\StringTok{ }\DecValTok{25}\NormalTok{) ^}\StringTok{ }\DecValTok{2}
\end{Highlighting}
\end{Shaded}

\begin{verbatim}
## [1] 272.25
\end{verbatim}

\begin{enumerate}
\def\labelenumi{\alph{enumi}.}
\setcounter{enumi}{1}
\item
\end{enumerate}

The sample size would need to be larger, in order to be more accurate.
This can be seen with the z score difference.

\begin{enumerate}
\def\labelenumi{\alph{enumi}.}
\setcounter{enumi}{2}
\item
\end{enumerate}

Z score because the sample size is \textgreater{} 30 and the SD is
known.

\begin{center}\rule{0.5\linewidth}{\linethickness}\end{center}

5.20

\begin{enumerate}
\def\labelenumi{\alph{enumi}.}
\item
\end{enumerate}

Reading score tends to be lower than writing score, but the difference
is not massive, as some students seem to be better in reading than
writing.

\begin{enumerate}
\def\labelenumi{\alph{enumi}.}
\setcounter{enumi}{1}
\item
\end{enumerate}

It is unlikely that they are completely independent from each other as
reading and writing are tied.

\begin{enumerate}
\def\labelenumi{\alph{enumi}.}
\setcounter{enumi}{2}
\item
\end{enumerate}

Null: there is no difference in reading and writing scores, ALT: there
is a difference in reading and writing scores

\begin{enumerate}
\def\labelenumi{\alph{enumi}.}
\setcounter{enumi}{3}
\item
\end{enumerate}

The samples should be independent, random, greater than 30, and have no
skew.

\begin{enumerate}
\def\labelenumi{\alph{enumi}.}
\setcounter{enumi}{4}
\item
\end{enumerate}

The pvalue shown below is not less than the needed .05. This means that
there is no difference between reading and writing scores.

\begin{Shaded}
\begin{Highlighting}[]
\NormalTok{diff <-}\StringTok{ }\NormalTok{-.}\DecValTok{545}
\NormalTok{sd <-}\StringTok{ }\FloatTok{8.887}
\NormalTok{n <-}\StringTok{ }\DecValTok{200}

\NormalTok{se <-}\StringTok{ }\NormalTok{sd /}\StringTok{ }\KeywordTok{sqrt}\NormalTok{(n)}
\NormalTok{t <-}\StringTok{ }\NormalTok{diff /}\StringTok{ }\NormalTok{se}
\NormalTok{df <-}\StringTok{ }\DecValTok{200} \NormalTok{-}\StringTok{ }\DecValTok{1}
\NormalTok{p <-}\StringTok{ }\KeywordTok{pt}\NormalTok{(t, }\DataTypeTok{df =} \NormalTok{df);p}
\end{Highlighting}
\end{Shaded}

\begin{verbatim}
## [1] 0.1934182
\end{verbatim}

\begin{enumerate}
\def\labelenumi{\alph{enumi}.}
\setcounter{enumi}{5}
\item
\end{enumerate}

The source I would imagine is from there being a link between reading
and writing scores, AKA they are not independent. In this case we would
have incorrectly accepted the null hypothesis.

\begin{enumerate}
\def\labelenumi{\alph{enumi}.}
\setcounter{enumi}{6}
\item
\end{enumerate}

It is very likely to include 0, is there is indeed no difference between
reading and writing scores.

5.32

Null: difference = 0, Alt: difference != 0

\begin{Shaded}
\begin{Highlighting}[]
\NormalTok{mean_a <-}\StringTok{ }\FloatTok{16.12}
\NormalTok{mean_m <-}\StringTok{ }\FloatTok{19.85}
\NormalTok{sd_a <-}\StringTok{ }\FloatTok{3.58}
\NormalTok{sd_m <-}\StringTok{ }\FloatTok{4.51}
\NormalTok{n <-}\StringTok{ }\DecValTok{26}
\NormalTok{df <-}\StringTok{ }\NormalTok{n -}\StringTok{ }\DecValTok{1}
\NormalTok{diff_means <-}\StringTok{ }\KeywordTok{abs}\NormalTok{(mean_a -}\StringTok{ }\NormalTok{mean_m)}
\NormalTok{se <-}\StringTok{ }\KeywordTok{sqrt}\NormalTok{((sd_a ^}\StringTok{ }\DecValTok{2} \NormalTok{/}\StringTok{ }\NormalTok{n) +}\StringTok{ }\NormalTok{(sd_m ^}\StringTok{ }\DecValTok{2} \NormalTok{/}\StringTok{ }\NormalTok{n))}
\NormalTok{t <-}\StringTok{ }\NormalTok{diff_means /}\StringTok{ }\NormalTok{se}
\NormalTok{p <-}\StringTok{ }\KeywordTok{pt}\NormalTok{(t, df); }\DecValTok{1} \NormalTok{-}\StringTok{ }\NormalTok{p}
\end{Highlighting}
\end{Shaded}

\begin{verbatim}
## [1] 0.001441807
\end{verbatim}

The p value is less than 0.05, so we reject the null hypothesis, and say
that the means fuel milage differs.

5.48

\begin{enumerate}
\def\labelenumi{\alph{enumi}.}
\item
\end{enumerate}

Null: there is no difference between all means, Alt: 1 or more means
differ

\begin{enumerate}
\def\labelenumi{\alph{enumi}.}
\setcounter{enumi}{1}
\item
\end{enumerate}

I assume that the samples are independent and random. The samples should
also be normal. In this case the HS and Bachelor's boxplots show a
significant number of outliers.

\begin{enumerate}
\def\labelenumi{\alph{enumi}.}
\setcounter{enumi}{2}
\item
\end{enumerate}

\begin{longtable}[c]{@{}llllll@{}}
\toprule
ANOVA & Df & Sum Sq & Mean Sq & F value &
Pr(\textgreater{}F)\tabularnewline
\midrule
\endhead
degree & 4 & 2006.16 & 501.54 & 2.188984 & 0.0682\tabularnewline
Residuals & 1167 & 267,382 & 229.12 & &\tabularnewline
Total & 1171 & 269388.16 & & &\tabularnewline
\bottomrule
\end{longtable}

\end{document}
